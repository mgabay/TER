\section{Intoduction}
Ce document est un bref résumé des mes impressions sur les logiciels
et API utilisés au cours de mon TER.\\


\section{Produits IBM - ILOG}
\verb?ILOG? (qui a été racheté par \verb?IBM?) propose des solveurs
très puissants pour la programmation par contrainte et
la programmation linéaire en nombres entiers.\\
En tant qu'académique vous pouvez accéder gratuitement à tous leurs produits.
La page web du programme d'\verb?IBM? est \url{https://www.ibm.com/developerworks/university/}
et les logiciels sont accessibles à partir de
\url{https://www.ibm.com/developerworks/university/software/get_software.html}.\\

Tous les logiciels d'\verb?IBM? sont fournis avec des interfaces permettant de les
utiliser directement dans des programmes en \verb?C++? ou en \verb?Java?.
Ils sont également disponibles et optimisés pour de très nombreuses architectures de machines.

\subsection{CPLEX}
\verb?CPLEX? est un solveur permettant de résoudre des programmes linéaires en nombres entiers.
Il est très puissants et relativement rapide à prendre en main dans son fonctionnement de base.
Toutefois, il dispose de (très) nombreuses options.\\
Je pense que sa puissance réside
également dans ses options mais je n'ai utilisé pour le TER que le ``minimum vital'' de ce point de vue.
Une lecture approfondie du manuel pourra se révéler très interessante pour comprendre celles-ci
ainsi que le fonctionnement interne du solveur.\\

Il est assez difficile d'estimer le temps nécessaire à la résolution d'un problème par \verb?CPLEX? ;
en particulier car le solveur peut passer une très grande partie du temps de calcul à prouver une solution
trouvée très rapidement.
\verb?CPLEX? se comporte également parfois de manière obscure (une contrainte spécifié avant ou après une autre peut
radicalement changer le temps de calcul).\\\\
J'ai utilisé les versions \verb?10? et \verb?12? du logiciel et j'ai remarqué des différences notables :
\begin{itemize}
 \item J'ai constaté une nette amélioration générale des performances dans \verb?CPLEX? \verb?12?.
 \item \verb?CPLEX? \verb?12? semble prouver beaucoup plus vite les solutions.
 \item La gestion des contraintes a également, à mon avis, radicalement changé entre ces deux versions :
    j'ai constaté ce changement en résolvant plusieurs problèmes avec deux modèles différents - l'un
    correspondant au deuxième dans lequel on a dupliqué une contrainte pour la spécifier explicitement.
    Lors de la résolution par \verb?CPLEX? \verb?10?, les deux modèles étaient équivalents (même temps de calcul,
    même nombre de noeuds explorés, comptés, restants...) alors que dans la version \verb?12? on observe une différence considérable
    entre les deux modèles.
\end{itemize}

\verb?CPLEX? est également difficile a utiliser sans un langage de modélisation. À ce titre, \verb?OPL?
est un candidat parfait. Je reviendrai dessus un peu plus loin.\\

\verb?CPLEX? est, à mon avis, un excellent solveur, toutefois je pense qu'il est nécessaire
d'étudier ses options si l'on veut véritablement exploiter tout son potentiel.\\
Il peut être également intéressant de tester les performances de \verb?CPLEX? contre \verb?CP? ;
il est possible que certains PLNE soient résolus beaucoup plus rapidement
si on utilise la programmation par contraintes.

\subsection{CP}
\verb?CP? est le solveur d'\verb?IBM? - \verb?ILOG? pour la programmation par contrainte.
J'ai utilisé la version 2 et ai tout simplement été bluffé par ses performances (il y a un
facteur $> 10$ par rapport à choco sur les cas testés et les temps de calcul ne sont même
pas comparables avec \verb?CPLEX?).\\

Comme \verb?CPLEX? il est à utiliser avec un langage de modélisation, comme \verb?OPL?.
En revanche, je n'ai pas regardé ses options.\\

Je serais très intéressé de connaître les résultats de tests des performances approfondis de \verb?CP?
contre \verb?CPLEX?. Les exemples que j'ai traité donnant très largement l'avantage à \verb?CP?...

\subsection{OPL}
\verb?OPL? est le langage de modélisation d'\verb?IBM? - \verb?ILOG?. Il est très intuitif est
très simple à prendre en main. Il est fournit avec une \verb?IDE? pour sa version windows
(je ne l'ai pas testé).\\
L'exécutable d'\verb?OPL? permet de résoudre directement les problèmes sans avoir à les
exporter préalablement pour les soumettre au solveur ensuite (chose qu'il permet également).\\

D'après moi, \verb?OPL? est tout simplement le langage ``naturel'' pour résoudre des
problèmes avec \verb?CP? ou \verb?CPLEX?.

\section{Logiciels libres}
J'ai également utilisé plusieurs logiciels libres au cours de mon TER.
Ils sont tous très intéressants dans leurs domaines.

\subsection{Boost}
\verb?Boost?\lien{http://www.boost.org/} est une librairie très riche dont j'ai utilisé la partie graphe.\\
C'est une librairie \verb?C++? implémentée entièrement à base de templates
ce qui lui octroie une grande flexibilité. Toutefois cette flexibilité est coûteuse car
la compilation des template est très longue ce qui ralentit le processus de développement.
Pour palier à celà, \verb?Boost? dispose de bindings \verb?Python? permettant de s'affranchir
de l'étape de compilation des programmes en programmant en \verb?Python? plutôt qu'en \verb?C++?.\\
\verb?Boost? différencie concepts (\verb?graph_traits?) et implémentation pour un graphe. Cette 
différence est assez subtile et rend \verb?Boost? extrèmement flexible mais très
difficile à manipuler. Pour une utilisation intensive, la lecture d'un livre de référence
s'impose et la documentation ne suffira probablement pas.\\

\verb?Boost? dispose de très nombreux algorithmes classiques (exploration de graphes,
plus courts chemins, composantes connexes, flot max, ...).
Toutefois il ne propose pas d'algorithmes pour les problèmes $NP-durs$, bien que quelques
heuristiques sont données à titre d'exemples dans la documentation.\\

\verb?Boost? est très puissant et complet. Cette librairie est optimisée et destinée
a être utilisé dans des programmes en \verb?C++?, ce qui permet d'obtenir du code
très performant. Toutefois, son utilisation n'est pas toujours naturelle même pour des
programmeurs expérimentés comme en témoignent les forums.

\subsection{Choco}
\verb?Choco?\lien{http://www.emn.fr/z-info/choco-solver/index.html} est un projet français de
solveur pour la programmation par contraintes.
C'est une librarie \verb?Java? très simple à manipuler.\\

La force de \verb?Choco? réside à la fois dans sa simplicité (très simple à utiliser,
dans un langage de programmation facilement accessible) et ses performances
qui sont à mon goût très satisfaisantes même si elles sont bien inférieures à celles de \verb?CP?
qui doit, vraisemblablement, disposer de toutes les optimisations réalisées dans \verb?Choco?
puisque ce dernier est distribué sous licence BSD, une des moins restrictives du monde logiciel libre.\\

\verb?Choco? dispose également d'options avancées comme la possibilité de maximiser/minimiser une
fonction objective ou de chercher une solution ou toutes les solutions à un problème.\\

Toutefois, ce qui fait la force de \verb?Choco? fait aussi sa faiblesse : le choix de \verb?Java?
comme langage de programmation ne permet pas d'espérer pouvoir atteindre des performances maximales.\\

Malgré ce dernier défaut, \verb?Choco? est un projet extrèmement intéressant et n'ayant pas à
pâlir de ses performances. Des tests de performances se révèlerait intéressants pour lever le voile
sur la différence avec son concurrent \verb?CP?. Même si l'issue ne fait pas de doûte, l'estimation
du facteur peut se révéler des plus instructives.

\subsection{Cliquer}
\verb?Cliquer?\lien{http://users.tkk.fi/pat/cliquer.html} est un projet de Patric R. J. Östergård's\lien{http://users.tkk.fi/~pat/}.
Il propose un algorithme exact pour trouver une clique max dans un graphe quelconque (fichier
fournit au format \verb?Dimacs?) ainsi qu'une API permettant d'utiliser ses programmes dans
des programmes en \verb?C/C++?. Le programme principal est très simple à utiliser
dans la ligne de commande.\\

L'efficacité de son programme est très impressionnante (de l'ordre du dixième de secondes pour
trouver une clique max dans un graphe à plusieurs centaines de sommets et arêtes) mais ceci
reste à confirmer par d'autres tests puisqu'il semblerait que dans les fichiers que j'ai testé
les sommets de la clique max sont en tête.


\section{Conclusion}
Tous ces programmes sont intéressants dans leurs domaines et présentent des points forts comme des points faibles.
Je vous ai fait part de mes impressions tout en essayant d'être le plus objectif
possible mais je n'ai eu qu'un aperçu de la plupart des ces programmes et
tout ce qui concerne les performances reste à
confirmer par de véritables benchmarks.\\

J'espère que ces quelques lignes vous auront été utiles :)