\pagenumbering{arabic}
\setcounter{page}{1}

\section*{Remerciements}
\addcontentsline{toc}{section}{Remerciements}
Avant toute chose, je tiens à remercier Nadia \textsc{Brauner} pour son encadrement et ses conseils
éclairés tout au long du TER et Julien \textsc{Dardel}, notamment,
pour l'aide qu'il m'a apporté avec \verb?CPLEX?
et pour trouver des instances de test.

\section{Introduction}
La question posée par le sujet est la suivante :\\
\textbf{Quelle taille de graphe peut-on colorier optimalement avec un ordinateur aujourd'hui ?}\\

Colorier un graphe consiste à affecter une couleur à chacun de ses
sommets de sortes que deux sommets adjacents n'aient pas la même couleur (on peut de la même façon définir
la coloration des arêtes mais ce problème se ramène très failement à la coloration des sommets en envisageant
le graphe de lignes correspondant).\\
Le problème consistant à déterminer le nombre chromatique d'un graphe $G$
(i.e. le nombre minimal de couleurs nécessaires, noté $\chi(G)$)
est $\mathcal{NP}-complet$ c'est donc un problème difficile pour lequel trouver la solution exacte en
un temps polynomial avec un algorithme déterministe est impossible à moins que $\mathcal{P}=\mathcal{NP}$.\\

L'objet de ce TER est donc de déterminer - compte-tenu des outils et ressources disponibles aujourd'hui -
jusqu'à quelle taille de graphe on peut espérer pouvoir le colorier optimalement en un temps raisonnable.\\

Afin d'y répondre, j'ai commencé par me documenter puis à envisager plusieurs solutions pour la coloration de graphes.
J'ai ensuite implémenté une solution sous forme de programme linéaire en nombre entiers et j'ai utilisé \verb?CPLEX?
pour résoudre des instances du problème. Puis, j'ai amélioré mon modèle en ajoutant des coupes intéressantes.
Après ceci, j'ai modélisé le problème en utilisant la programmation par contrainte et je l'ai
résolu avec \verb?Choco? et \verb?CP?.\\
Enfin, j'ai soumis mes algorithmes à une campagne de tests dont j'ai interprété les résultats.\\

Dans mon processus de réflexion, je me suis focalisé sur des graphes simples et connexes, toutefois
j'ai pris soin de faire en sorte que mes modèles soient valides pour des multigraphes sans surcoût
(on considère alors le graphe simple correspondant au multigraphe) et des graphes non connexes.
Toutefois, dans le cas de graphes non connexes il serait plus judicieux de passer par une étape
de préprocessing identifiant et séparant les composantes connexes pour les soumettre séparément
au solveur plutôt que de soumettre le graphe directement.

\section{Bibliographie}
Comme indiqué dans l'introduction, le première partie de mon travail a été un travail bibliographique.
J'ai lu les chapitres concernant la coloration de graphe (coloration des arêtes \cite{berge-arretes} et
des sommets \cite{berge-sommets}) du livre Graphes et Hypergraphes \cite{berge-graphes} de Claude Berge.
Je me suis également intéréssé au livre Graph coloring problems \cite{jensen1996graph} de Jensen et Toft.\\
Bien entendu, je me suis également servi d'internet et de \href{http://en.wikipedia.org/wiki/Main_Page}{wikipedia}
\cite{wiki}.\\

De cette étude sont ressortis plusieurs résultats et en particulier la minoration suivante :
\begin{theorem}
\label{minor}
Dans un graphe simple $G$ de $n$ sommets et $m$ arêtes, on a :
$$\chi(G) \geq \frac{n^2}{n^2-2m}$$
\end{theorem}
Qui donne une borne inférieure très intéressante, notamment lorsque le graphe est dense.

\section{La coloration comme Programme Linéaire en Nombres en Entiers}
\subsection{Modélisation}
La première partie de mon travail a été de modéliser le problème de coloration de graphe
en programme linéaire en nombres entiers.\\
Le modèle obtenu et utilisé est le suivant :\\
Pour un graphe $G=(V,E)$ :
\begin{equation}
\label{model-plne}
\left\{
    \begin{array}{ll}
        \text{min }z & \\
	c_i \leq z & \forall i \in V\\
        c_i \leq c_j - 1 + n\times y_{i,j}& \forall (i,j)\in E\\
        c_j \leq c_i - 1 + n\times (1-y_{i,j})& \forall (i,j)\in E\\
	y_{i,j} \in \{0,1\} & \forall (i,j)\in E\\
	c \in (\mathbb{N}^*)^{|V|}
    \end{array}
\right.
\end{equation}

$c$ correspond au vecteur de la coloration et $y$ à l'activation des contrainte (en fait
$y$ permet de linéariser la contrainte $|c_i-c_j| \geq \delta_{i,j}$ où $\delta_{i,j}=1$
si $(i,j)\in E$ et $\delta_{i,j}=0$ sinon - dit autrement, cette contrainte correspond à
``deux sommets adjacents ne sont pas de même couleur'').\\

\subsection{Résolution}
Pour modéliser et résoudre ce problème, j'ai utilisé \verb?CPLEX? et \verb?OPL?.\\
\verb?CPLEX? est un solveur permettant de résoudre des programmes linéaires en nombres entiers
de manière exacte. C'est un logiciel d'\verb?IBM?-\verb?ILOG? \cite{ibm}.\\
\verb?OPL? est le langage de modélisation d'\verb?IBM?-\verb?ILOG?. Il permet de
modéliser facilement des problèmes complexes pour les soumettre à \verb?CPLEX?.\\

Toutefois, la programmation linéaire en nombre entier est un problème $\mathcal{NP}-complet$,
il ne faut donc pas espérer pouvoir résoudre facilement le programme linéaire en nombres
entiers (\ref{model-plne}) facilement pour n'importe quel graphe.\\

Afin de pouvoir résoudre le problème, je l'ai modélisé et j'ai recherché 
des majorations et des contraintes permettant de résoudre plus rapidement le problème.
J'ai également essayé d'optimiser l'aspect algorithmique du problème en évitant des opérations inutiles
(placement des contraintes, initialisation et ajout de contraintes sous certaines conditions...).\\
J'ai pris le soin de démontrer les coupes utilisées et de vérifier leurs compatibilités afin
de garantir mon modèle.\\

Pour représenter les graphes, j'ai utilisé la matrice d'adjacence du graphe car compte-tenu de la
complexité du problème, celle-ci sera toujours de dimension réduite. De plus, c'est l'outil idéal
pour manipuler les données car le problème de coloration de graphe est un problème d'adjacence des
sommets (deux \underline{sommets adjacents} ne sont pas de même couleur).\\

J'ai modélisé $y$ comme une matrice de $\mathcal{M}_n(\{0,1\})$ où $n$ est l'ordre du graphe
à colorier (i.e. le nombre de sommets du graphe).\\

\underline{Remarque} : j'ai fait en sorte que les programmes et modèles permettent de colorer des
multigraphes (les graphes simples étant des multigraphes particuliers).

\subsubsection{Validation}
A chaque amélioration du modèle, je l'ai testé sur de nombreux graphes afin de
permettre d'une part de valider l'interêt des améliorations apportées en termes d'efficacité de la résolution
et d'autre part,
de valider le modèle (vérifier que le nombre de couleurs obtenu est bien le nombre
chromatique et que la coloration optimale renvoyée est valide).\\

Pour vérifier la bonne coloration du graphe, j'ai testé les modèles sur des graphes
de petite taille et je les ai vérifiés manuellement.\\
De plus, afin d'effectuer les tests d'efficacité et de valider mes modèles, j'ai choisit dans un premier
temps d'utiliser comme instances de tests des graphes complets.
J'ai fait ce choix pour deux raisons. La première est qu'ils me permettent de vérifier
facilement la bonne coloration.
\hypertarget{maxc}{La} seconde est qu'ils constituent le pire cas pour le nombre
chromatique (celui-ci est majoré par $d+1$ où $d$ est le degré maximum d'un sommet dans le graphe
simple correspondant et dans le cas d'un graphe complet, $\chi(G)=d+1$), dans le cas où on ne
dispose pas d'une minoration intéressante ils forcent donc le solveur a énumérer et tester un très grand
nombre de solutions.\\

Toutefois, une minoration interessante est fournie pas le théorème \ref{minor} (en particulier, elle vaut
$n$ pour un graphe complet d'ordre $n$ : $K_n$). Une fois cette minoration
implémentée, j'ai donc du tester mes modèles sur d'autres exemples.\\
J'ai choisit d'utiliser les instances du challenge dimacs de coloration de graphe \cite{dimacs}, celles-ci offrant des
exemples variés et dont on connait, pour une partie, le nombre chromatique.
J'ai également testé l'efficacité de la résolution sur des multigraphes connexes que j'ai généré en utilisant
la librairie \verb?Boost?. La répartition des arêtes suivant une loi uniforme.\\

Les instances utilisées sont disponibles aux adresses suivantes (fichiers .col), sur
la page de Michael Trick \cite{tepper} :
\url{http://mat.tepper.cmu.edu/COLOR04/} et \url{http://mat.gsia.cmu.edu/COLOR/instances.html}.\\
\verb+Boost+ \cite{Boost} est une collection de librairies \verb?C++? open-source. Vous trouverez plus
d'informations sur \verb+Boost+ à cette adresse : \url{http://www.boost.org/}. J'ai particulièrement utilisée
la \verb?Boost Graph Library? \cite{bgl}.

\subsubsection{Optimisations}
\label{opt}
Comme expliqué précédemment, j'ai amélioré et testé les performances de mes algorithmes.
Je vais ici détaillé les améliorations apportées aisi que les gains associés en performances
(donnés en pourcents - ces gains correspondent à la moyenne de gains de performances par rapport au modèle
précédent sur des graphs testés dans les deux cas).\\
Pour tester ces exemples, j'ai utilisé \verb?CPLEX? 10 pour architectures 32 et 64 bits afin
de tester les performances du solveur également. Mes tests m'ont rapidement poussé à adopter
\verb?CPLEX? 10 pour architectures 64 bits, le gain en performances étant non négligeable.\\

\underline{Remarque} : afin de vous faire ressentir les améliorations apportées en termes de
performances, j'indique dans la suite de cette partie certains temps écoulés pour la coloration et le
nombre d'itérations de l'algorithme.
Les données en temps ne sont qu'un indicateur subjectif : elles dépendent fortement de la machine,
du nombre d'itérations de l'algorithme et du nombre de noeuds explorés lors de la résolution.
Les calculs ayant été effectués sur mon ordinateur
portable, les données en temps ne sont en aucun cas représentative de ce qui aurait été obtenue
sur un serveur dédié. Des tests de performance plus poussés et détaillés se trouvent
dans la section \ref{tests}.\\

Le modèle initial ne correspond pas exactement au modèle (\ref{model-plne}) - j'y ai apporté la
modification expliquée \hyperlink{maxc}{plus haut} : le vecteur des couleurs est un
vecteur d'entiers compris entre 1 et $d+1$ (où $d$ est le degré maximum d'un sommet du graphe simple
correspondant). Pour ce premier modèle, les performances sont exécrables : l'algorithme
colore $K_{10}$ en 52 secondes et 428761 itérations de l'algorithme
($K_n$ est le graphe complet à $n$ sommets).\\

Les amélioration apportées sont les suivantes (le gain est le pourcentage
de temps économisé par rapport à la résolution utilisant toutes les optimisations précédentes) :
\begin{enumerate}
 \item \label{p1}\textbf{Fixer la couleur du premier sommet} (\verb?c[1] == 1? dans le modèle).\\
    Coloration de $K_{10}$ en 3,8 secondes et 30470 itérations de l'algorithme.\\
    Coloration de $K_{30}$ non terminée après 2 heures et presque 15 millions d'itérations.\\
    \textbf{Gain} : 93\%.
 \item \label{p2}\textbf{Ajout de la minoration obtenue par le théorème \ref{minor}} (\verb?z >= chi_min?).\\
    Cette minoration du nombre chromatique est particulièrement efficace pour les graphes très denses.\\
    En témoigne l'amélioration des performances sur les graphes complets :\\
    Coloration de $K_{10}$ en moins de 0,1 secondes et 72 itérations de l'algorithme.\\
    Coloration de $K_{30}$ en 1,2 secondes et 2275 itérations de l'algorithme.\\
    Coloration de $K_{100}$ en 36 secondes et 34645 itérations de l'algorithme.\\
    \textbf{Gain} : 
    \begin{itemize}
      \item $>99\%$ sur les graphes complets (car on a alors
	\verb?chi_min?$=n=\chi(G)$).
      \item élevé sur les graphes denses.
      \item faible sur les graphes peu denses
    \end{itemize}
 \item \label{p3}\textbf{Initialisation de \verb?y[1][i]? et \verb?y[i][1]?}.\\
    On avait précédemment rajouté l'initialisation du premier sommet, on peut
    également initialiser les valeurs de $y$ correspondantes.\\
    \textbf{Gain} : presque nul sous \verb?CPLEX? 10 car 1 étant le premier sommet, \verb?CPLEX? 10 fixe
      les valeurs de $y$ correspondantes au début de chaque itérations avant de passer aux autres sommets.
      C'est en quelque sorte une duplication de contrainte.\\
      Toutefois, j'ai constaté un gain de performances de l'ordre de $5\%$ sous \verb?CPLEX? 12 pour une duplication
      similaire. C'est probablement parce qu'il fixe alors définitivement les valeurs de $y$ correspondantes.
 \item \label{p4}\textbf{Test de la valeur de \verb?m[i][j]? pour ajouter les contraintes de non adjacence des couleurs}.\\
      \verb?m? est la matrice d'adjacence du graphe.\\
      Jusqu'à présent, les contraintes de non adjacence des couleurs (les deux dernières inéquations dans
      \ref{model-plne}) étaient de la forme :\\
      \verb?m[i][j] * c[i] <= c[j] - 1 + V * y[i][j]?,
      on avait donc dans le modèle un nombre important de contraintes inutiles et pénalisant l'exécution de l'algorithme (car
      elles doivent quand même être évaluées pour fixer les valeurs de $y$). De plus, le modèle ne fonctionnait
      alors que pour des graphes \textbf{simples}.\\
      En ajoutant la contrainte \verb?c[i] <= c[j] - 1 + V * y[i][j]?
      si et seulement si \verb?m[i][j] != 0? (c'est le test), on élimine des contraintes pénalisant l'algorithme
      et on peut désormais colorier des \textbf{multigraphes} - en prennant soin de bien calculer les autres variables :
      par exemple, $d$, le degré maximum d'un sommet utilisé pour certaines majorations doit être
      celui du graphe simple correspondant, il n'est donc pas égal à
      $\frac{1}{2}\underset{i}{\max}(\sum_{j=1}^n m_{i,j})$ mais à
      $\frac{1}{2}\underset{i}{\max}(\sum_{j=1}^n \mathds{1}_{m_{i,j}\neq0})$,
      il en va de même pour les autres variables qui sont à calculer dans
      le graphe simple correspondant.\\
      \textbf{Gain} : $14\%$ sur $K_{100}$ (colorié en 31s).   
 \item \textbf{Ajout des contraintes de non adjacence des couleurs pour $j > i$ seulement}.\\
      Il s'agit en fait d'une contrainte inutilement dupliquée car on a :
      $$c_i \leq c_j - 1 + n\times y_{i,j} \Leftrightarrow c_i \leq c_j - 1 + n(1 - \times y_{j,i})$$
      et de même :
      $$c_j \leq c_i - 1 + n\times (1-y_{i,j}) \Leftrightarrow c_j \leq c_i - 1 + n\times y_{j,i}$$
      car $y_{i,j}=1-y_{j,i}$.\\
      Il suffit donc d'écrire ces contraintes une seule fois.\\
      Ceci réduit le nombre de contraintes total de plus de la moitié (on passe de $2n^2$ contraintes de
      non adjacence des couleurs à  $n^2 - n$ contraintes).
 \item \label{p5}\textbf{Teste si le graphe est complet} (\verb?chi_min == n?).\\
    C'est l'ajout d'un cas particulier coloriable trivialement.\\
    \textbf{Gain} : $>99\%$ si le graphe est complet, $0\%$ sinon (pas de pertes
    de performances).
 \item \label{p6}\textbf{Ajout de la contrainte $y_{i,j}=1-y_{j,i}$}.\\
      C'est l'interêt de $y_{i,j}$ qui permettent l'activation des contraintes des couleurs différentes.\\
      Il suffit de remplacer $y_{i,j}$ par sa valeur dans le modèle (\ref{model-plne}) pour constater que dans le cas où
      $y_{i,j}=0$, la première inéquation appliquée à $i$ et $j$ donne $c_i\leq c_j - 1$ et $c_j \leq c_i - 1$
      ce qui est absurde. De même avec la deuxième équation si $y_{i,j}=1$.\\
      Notons que lorsque cette contrainte est mal placée dans le programme (i.e. après les contraintes les
      deux inéquations), l'effet est désastreux et on perd fortement en efficacité.\\
      \textbf{Gain} : $46\%$ sur des multigraphes dont les arêtes sont distribuées selon
      une loi uniforme.
 \item \label{p7}\textbf{Ajout de la contrainte $c_i \leq i$}.\\
    Il existe toujours une coloration optimale telle que pour tout $i$, $c_i \leq i$.
    En ajoutant cette contrainte, on limite grandement le nombre de colorations que
    le solveur va tester. Même si on limite également le nombre de solutions avec une coloration
    optimale qu'il peut trouver, cette diminution est négligeable devant celle du nombre
    de noeuds explorés par le solveur.
    \textbf{Gain} : $87\%$ sur des multigraphes dont les arêtes sont distribuées selon
      une loi uniforme.
\item \label{p8}\textbf{Prise en compte d'une clique dans la coloration}.\\
    Une clique est un sous-graphe complet du graphe d'origine.\\
    Les sommets d'une clique étant tous adjacents entre eux, ils sont deux à deux de couleurs
    différentes. On peut donc, pour initialiser la résolution, commencer par affecter
    des couleurs aux sommets d'une clique.\\
    Toutefois, si on veut préserver la contrainte très puissante : $c_i\leq i$, il faut prendre
    le soin de mettre les $k$ sommets de la clique en positions 1 à $k$ de la matrice d'adjacence
    afin de pouvoir initialiser leurs couleurs en garantissant qu'il existera toujours une coloration
    optimale avec $c_i\leq i$.\\
    \textbf{Gain} : le gain dépend fortement des caractéristiques du graphe. Plus la taille de la
    clique est proche du nombre chromatique, plus le gain sera substantiel (celui-ci allant de $0$
    à $70\%$).
\end{enumerate}

J'ai également essayé d'ajouter la contrainte $c_i\leq d(v_i)+1$ ($d(v)$ est le degré de $v$)
mais celle-ci ne permet pas de gagner
en performances. Dans certains cas, elle dégrade même les performances de la résolution.\\

Toutes ces optimisations ont permis de passer d'un modèle exécrable à un modèle permettant
d'envisager de colorier optimalement bon nombre de graphes.\\
Vous trouverez section \ref{tests} des tests de performances et des analyses. 
 
\section{La coloration comme Programme Par Contraintes}
La programmation par contrainte est un moyen puissant permettant de décrire et résoudre des
problèmes. Elle consiste à décrire le problème en utilisant des contraintes.
Contrairement à la programmation linéaire en nombre entier, nous ne sommes pas
limités à des inéquations mais nous pouvons également utiliser toute sortes d'outils
tels que les inégalités, les minimum, les maximum, des valeurs absolues...\\
La puissance de la programmation par contrainte réside dans sa simplicité
comme en témoigne la modélisation du problème de coloration de graphe section suivante.\\

Tout comme la résolution d'un programme linéaire en nombre entier,
la résolution d'un programme par contraintes est un problème $\mathcal{NP}-complet$.
Toutefois, les méthodes employées pour résoudre un problème par contrainte
diffèrent de la programmation linéaire en nombre entiers.
Employer une deuxième modélisation, permettra donc de voir quelle méthode
se prête le mieux à la résolution du problème de coloration de graphes.

\subsection{Modélisation}
Le problème de coloration de graphe se modélise en programmation par contraintes
de la manière suivante :
\begin{equation}
\label{model-ppc}
\left\{
    \begin{array}{ll}
        \text{min }z & \\
	c_i \leq z & \forall i \in V\\
        c_i \neq c_j & \forall (i,j)\in E\\
	c \in (\mathbb{N}^*)^{|V|}
    \end{array}
\right.
\end{equation}


\subsection{Résolution du programme par contraintes}
Afin de résoudre le problème, j'ai utilisé \verb?CHOCO? 2.1.1 \cite{choco}
et \verb?CP? 2.\\
Le premier est un solveur open-source dévelopé, entre autres, par des chercheur nantais.
Il offre une API en \verb?Java?.\\
Le second est développé par \verb?IBM? - \verb?ILOG? \cite{ibm}, tout comme \verb?CPLEX?.\\

L'utilisation de ces deux solveurs m'a permis de tester leurs performances, l'un contre l'autre
et contre la programmation linéaire en nombre entiers.\\

Pour représenter les graphes, j'ai utilisé la matrice d'adjacence.\\

\underline{Remarque} : sous \verb?CHOCO?, j'ai remplacé la contrainte $c_i \leq z$ $\forall i \in V$
 par $z = \underset{v_i\in V}{\max} (c_i)$.
Des tests que j'ai effectué avec \verb?CP? montrent que la contrainte $c_i \leq z$ $\forall i \in V$
est en fait très légèrement meilleure (on explore quelques noeuds de moins pour la résolution).

\subsubsection{Optimisations}
De même que précédemment, le modèle peut être amélioré.\\
J'ai choisit d'optimiser celui-ci dans le programme pour \verb?CP? car ce solveur se révèle plus
performant que \verb?CHOCO? comme nous le verrons plus loin (section \ref{tests}).\\

J'y ai introduit toutes les optimisations vu précédemment (cf sous-section \ref{opt})
n'ayant pas trait au programme linéaire, ce qui correspond aux points : \ref{p1},
\ref{p2}, \ref{p3}, \ref{p4}, \ref{p5}, \ref{p6}, \ref{p7} et \ref{p8}.

\section{Programmes et outils}
\subsection{Programmes utilisés}
Comme expliqué plus haut, j'ai utilisé \verb?OPL? (modélisation), \verb?CPLEX? (programmation linéaire en nombres entiers)
et \verb?CP? (programmation par contraintes)
de \verb?IBM - ILOG? \cite{ibm} pour modéliser et
résoudre les problèmes.\\
J'ai également utilisé \verb?CHOCO? \cite{choco} pour la programmation par contraintes.\\

Outre les solveurs, j'ai utilisé \verb?nauty? \cite{nauty} de Brendan D. McKay pour générer quelques graphes (\verb?nauty?
permet d'énumérer des graphes non isomorphes). Toutefois, celui-ci énumérant tous les graphes, il devient inutilisable
dès qu'on veut obtenir des graphes ayant plus de 10 sommets. J'ai donc écrit des programmes me permettant de
générer des graphes. Vous trouverez plus de détails à leurs propos dans la section suivante.\\

Pour trouver une clique maximum dans un graphe, j'ai utilisé \verb?Cliquer? \cite{cliquer} de
Sampo Niskanen et Patric R. J. Östergård. Ce programme m'a permis d'évaluer les performances
de mon heuristique et s'il était envisageable de chercher la clique max dans un graphe pour
colorier celui-ci (pour ceci, il faudrait que le coût pour la recherche d'une clique max soit
négligeable devant le coût de la coloration).

\subsection{Mes programmes}
J'ai écrit les modèles pour \verb?CPLEX? et \verb?CP? en \verb?OPL? (c'est le langage de modélisation de
\verb?IBM - ILOG? \cite{ibm}).\\

Comme expliqué plus haut, j'avais besoin d'un outil me permettant de générer des graphes.
J'ai donc utilisé la \verb?Boost Graph Library? \cite{bgl} pour écrire des programmes (en \verb?C++?)
me permettant d'importer des graphes du challenge Dimacs \cite{dimacs} de coloration et de générer des multigraphes
connexes dont la répartition des arêtes suit une loi uniforme.\\

Afin de visualiser ces graphes (soit générés, soit importés), j'ai également introduit une option
dans mes programmes qui permet de générer un fichier correspondant au graphe
au format \verb?dot? pour \verb?graphviz?.


J'ai également codé une heuristique très basique de recherche de clique dans un graphe, l'objectif étant
de démontrer l'interêt où non de prendre en compte une clique pour la coloration.
L'heuristique est sans garantie. Elle prend le sommet de plus grand degré et son voisin
de plus grand degré puis cherche le sommet voisin des deux premiers de plus grand degré et ainsi de suite.\\
Afin de pouvoir prendre en compte cette clique, mon programme réordonne ensuite les sommets pour que
ceux de la clique soient en tête de la matrice d'adjacence.\\

Pour résoudre un problème, trois possibilités se présentent donc :
\begin{enumerate}
 \item on dispose d'un fichier au format dimacs pour la coloration de graphes.\\
    Il suffit alors de le soumettre au parser qui génère un fichier qu'on peut soumettre au solveur.
 \item on génère aléatoirement un graphe.\\
    Le fichier généré est alors au format adéquate.
 \item on dispose de la matrice d'adjacence du graphe.\\
    Il suffit de la mettre en forme pour pouvoir soumettre le graphe au solveur.
\end{enumerate}


Tests : limitations temps / ressources (au labo vieux CPLEX / CP et Gantt chargé)
diff CPLEX 10 / 12 symétries

CPLEX 12 moins bonnes perfs que 10 sur graphes complets, premiers modèles


Le modèle doit être lancé avec un fichier contenant :
le nombre d'arêtes du graphe
 dans une première version
simple ne prennant en compte que le modèle et utilisant comme 
CPLEX 32 et 64 bit => tester l'outil
p\\\\
\section{Expérimentation / Analyse}
\label{tests}
Attention : CPLEX multicoeur pas CP
\section{Conclusion}
Perspectives : 1 h de CP = optimal non prouvé
dépends : densité + nb sommets/arêtes
cf ideas
clique de taille max + performante / meilleure heuristique
toutes les approches ne sont pas équivalentes !!! -> reliements contractions