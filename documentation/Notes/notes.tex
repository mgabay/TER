\section{Bibliographie}
\subsection{Définitions générales}
Graphe $G(X,E)$.

\begin{defin}[Ensemble d'articulation]
Si $G(X,E)$ est connexe et $A \in X$ est tel que $G_{X-A}$ n'est plus connexe,
alors $A$ est un ensemble d'articulation de $G$.
\end{defin}

\begin{defin}[Pièces]
Les composantes connexes $C_1,...C_p$ de $G_{X-A}$ définissents des graphes connexes :
$G_{C_1 \cup A},..., G_{C_p \cup A}$ appelés les pièces (relativement à $A$).
\end{defin}

\begin{defin}[Stable]
$S\in X$ est stable si deux sommets distincts de $S$ ne sont jamais adjacents, i.e. :
$\Gamma_G(S) \cap S = \varnothing$.
\end{defin}

\begin{defin}[Nombre de stabilité]
Soit $\mathcal{S}$ la famille des ensembles stables du graphe, le nombre de stabilité du graphe $G$ est :
$$\alpha(G)=\underset{S\in\mathcal{S}}{max}|S|$$
\end{defin}

\begin{defin}[Graphe planaire]
Un graphe $G$ est planaire s'il est possible de le représenter sur un plan de sorte que
les sommets soient des points distincts, les arêtes des courbes simples, et que deux arêtes
ne se rencontrent pas en dehors de leurs extrémités.
\end{defin}

\begin{defin}[Base d'un graphe]
$B$ est une base de $G(X,U)$ si :
\begin{enumerate}
 \item il n'existe pas de chemin reliant deux sommets distincts de $B$
 \item Tout sommet $x \notin B$ est l'extrémité initiale d'un chemin aboutissant dans $B$
\end{enumerate}
\end{defin}

\begin{defin}[Point d'articulation]
Sommet dont la suppression augmente le nombre de composantes connexes
\end{defin}

\begin{defin}[Isthme]
Arête dont la suppression augmente le nombre de composantes connexes
\end{defin}

\begin{defin}[Bloc]
Ensemble de sommets $A$ qui engendre un sous-graphe $G_A$ connexe, sans points d'articulation,
et maximal avec cette propriété.
\end{defin}

\begin{defin}[Graphe parfait]
Graphe tel que pour tout sous-graphe induit, le nombre chromatique est égal à la taille de la plus grande clique.
\end{defin}

\subsection{Etude générale}
Livre : cf \cite{berge-graphes}.
\subsubsection{Coloration des arêtes}
Référence : \cite{berge-arretes} + cf notes écrites
\subsubsection{Coloration des sommets}
Référence : \cite{berge-sommets}
\setcounter{defin}{0}

\begin{defin}[Nombre chromatique]
Plus petit nombre de couleurs nécessaire pour colorier les sommets, de 
sorte que deux sommets adjacents ne soient pas de même couleur.
Désigné par $\chi(G)$.
\end{defin}

\begin{defin}[Graphe k-chromatique]
Graphe tel que $\chi(G) \leq k$.
\end{defin}


\paragraph{Algorithmes pour la recherche du nombre chromatique}
\subparagraph{Principe des reliements-contractions :}
Soient $a$ et $b$ deux sommets de $G$ non adjacents,
on appelle \textit{reliement} le graphe $\tilde{G}=G \cup [a,b]$.
On appelle contraction le graphe $\bar{G}$ obtenu en assimilant
$\{ a,b \}$ à un sommet unique $c(a,b)$ joint à tout sommet $x \neq a,b$ de
$G$ adjacent à $a$ et/ou $b$.

Dans tout coloration de G, ou bien les sommets $a$ et $b$ ont la même couleur (et
alors c'est une coloration de $\bar G$) ou bien ils sont de couleurs différents (et
alors c'est une coloration de $\tilde G$).

En répétant le processus des reliements-contractions autant que possible pour $\bar G$
et $\tilde G$ considérés séparément, on finira par obtenir des graphes qui sont des cliques,
et pour lesquels le nombre chromatique est nécessairement égal au nombre de sommets.
Si la plus petit clique obtenue est une k-clique, on a : $\chi (G)=k$.

\subparagraph{Principe de la séparation des pièces :}
Si au cours de la procédure de reliements-contractions, on obtient un graphe admettant comme
ensemble d'articulation une clique $A$, la suppression de $A$ crée des composantes connexes
$C_1, C_2,\dots$; on remarque alors que les sous-graphes $G_{A \cup C_1}= H_1$, $G_{A \cup C_2}=H2,\dots$
peuvent être coloriés séparément à condition de faire coïncider les couleurs des sommets de $A$
dans chaque sous-graphe.

\begin{theorem}
Si G est un graphe d'ordre n, on a :
\begin{math}
\begin{cases}
\chi(G) * \alpha(G) & \geq n \\
\chi(G) + \alpha(G) & \leq n + 1
\end{cases}
\end{math}
\end{theorem}

\begin{theorem}[Gaddum, Nordhaus, 1960]
Si $\bar G$ est le graphe simple complémentaire du graphe simple $G$, on a :
$$\chi(G)+\chi(\bar G) \leq n+1$$
De plus, cette borne est la meilleur possible.
\end{theorem}

\begin{corollaire}
Si $\bar G$ est le graphe simple complémentaire d'un graphe simple $G$ d'ordre n, on a :
$$\chi(G)\chi(\bar G)\leq \left(\frac{n+1}{2}\right)^2$$
Cette borne est la meilleur possible.
\end{corollaire}

\begin{theorem}
Dans un graphe simple $G$ de $n$ sommets et $m$ arêtes, on a :
$$\chi(G) \geq \frac{n^2}{n^2-2m}$$
\end{theorem}

\begin{theorem}[Roy, 1967; Gallai, 1968]\label{Roy}
Etant donné un graphe simple $G=(X,E)$ avec $\chi(G)=q$, pour toute orientation des arêtes
il existe un chemin élémentaire de longueur $\geq q-1$ ; en outre, pour une certaine orientation,
il n'existera pas de chemin de longueur $> q-1$.
\end{theorem}

\begin{corollaire}
Si $G$ est un graphe simple coloré avec $q=\chi(G)$ couleurs $\alpha_1,\dots,\alpha_q$, il
existe une chaîne élémentaire qui rencontre successivement les $q$ couleurs $\alpha_i$
dans cet ordre.
\end{corollaire}
\setcounter{corollaire}{0}

\begin{rmq}
Le théorème \ref{Roy} implique l'existence d'un chemin hamiltonien dans un graphe complet.
\end{rmq}
\setcounter{rmq}{0}

\begin{theorem}
Si dans un graphe simple $G$, une partition $(S_1,S_2,\dots,S_q)$ est une $q$-coloration (pas
forcément minimale) et si l'on pose $d_k=\underset{x\in S_k}{max}\ d_G(x)$, alors :
$$\chi(G) \leq \underset{k \leq q}{max}\ min \{k, d_k+1\}$$ 
\end{theorem}

\begin{corollaire}
Soit $G$ un graphe simple dont les sommets sont indexés de sorte que $d_G(x_1)\geq d_G(x_2)\geq
\cdots\geq d_G(x_n)$.
Si, pour un entier $q \geq 0$, le nombre de sommets de degré $\geq q+1$ est $\leq q+1$ (et
en particulier, pour $q \leq n-2$, si $d_G(x_{q+2})\leq q$), alors on a $\chi(G) \leq q+1$.
\end{corollaire}

\begin{corollaire}
Si $G$ est un graphe simple de degré maximum $h$, on a $\chi(G)\leq h+1$.
\end{corollaire}
\setcounter{corollaire}{0}

\begin{rmq}[Utilisation du théorème 5 pour une meilleure borne sup]
Trouver une $k$-coloration $(S_1,\dots,S_k)$ telle que $S_1$ un ensemble stable
maximal qui contient beaucoup de sommets de degrés élevés, $S_2$ ensemble stable maximal
de $X-S_1$ qui contient le plus possible de sommets de degrés élevés...
\end{rmq}

\begin{rmq}[Gamme des degrés étendue]
Lorsque $\xi = |\{d_G(x) / x\in X\}|$ est élevé, cette méthode est intéressante :
$$\left [\left [\frac{\xi}{2}\right]\frac{1}{n-\xi} \right ]+1\leq \chi(G)\leq n-\left [\frac{\xi}{2}\right]$$
\end{rmq}
\setcounter{rmq}{0}

\begin{theorem}[brooks, 1941]
Soit $G$ un graphe simple de degré maximum $h$, qui n'admet pas pour composante
connexe une $(h+1)$-clique (ni, si $h=2$, un cycle impair); alors on a $\chi(G)\leq h$.
\end{theorem}

\subsubsection{Graphes $\chi$-critiques}
\setcounter{prop}{0}

\begin{defin}[Graphe $\chi$-critique]
Un graphe $G$ simple est dit $\chi$-critique si pour tout sommet $x_0$, le sous-graphe $G_0$
engendré par $X-\{x_0\}$ a un nombre chromatique $\chi(G_0) < \chi(G)$.
\end{defin}

\begin{prop}
Dans tout graphe $G$ avec $\chi(G)=q+1$, il existe un sous-graphe $\chi$-critique avec $\chi(G)=q+1$.
\end{prop}

\begin{prop}
Dans un graphe simple $\chi$-critique avec $\chi(G)=q+1$, le degré de chaque sommet $x$ vérifie :
$$d_G(x)\geq q$$
\end{prop}

\begin{prop}
Un graphe $\chi$-critique est connexe.
\end{prop}

\begin{prop}\label{crit-clique}
Un graphe $\chi$-critique n'admet pas une clique pour ensemble d'articulation.
\end{prop}

\begin{prop}
Un graphe $\chi$-critique n'admet pas de points d'articulation.
\end{prop}

\begin{prop}
Si $G$ est $\chi$-critique et avec $\chi(G)=q+1$ et si $A=\{a,b\}$ est un ensemble d'articulation
de deux éléments, il y a exactement deux pièces $B'_1 et B'_2$ relatives à cet ensemble d'articulation, et on a :
$$\chi(B'_1)=\chi(B'_2)=q$$
\end{prop}

\begin{prop}
SI $G$ est $\chi$-critique avec $\chi(G)=q+1\geq 4$, et si $A=\{a,b,c\}$ est un ensemble
d'articulation de trois éléments :
\begin{enumerate}
 \item S'il y a une arête dans $A$ : cet ensemble d'articulation admet au plus 3 pièces,
et celles-ci sont de nombre chromatique $q$.
 \item S'il y a deux arête dans $A$ : cet ensemble d'articulation admet au plus 2 pièces,
et celles-ci sont de nombre chromatique $q$.
 \item S'il y a trois arête dans $A$, $G$ n'est pas $\chi$-critique (prop \ref{crit-clique}).
 \item S'il n'y a pas d'arêtes dans $A$ : cet ensemble d'articulation admet au plus 5 pièces,
et celles-ci sont de nombre chromatique $q$ ou $q-1$.
\end{enumerate}
\end{prop}

\begin{prop}
Un graphe $G=(X,E)$, $\chi$-critique, avec $\chi(G)=q+1$, ne peut être disconnecté par l'élimination
de $q-1$ arêtes ; autrement dit :
$$m_G(A,X-A)\geq q\ \ \ \ (A\in X, A \neq \varnothing,X).$$
\end{prop}

\begin{theorem}
Si $G$ est un graphe $\chi$-critique avec $\chi(G)=q+1$, on a $d_G(x)\geq q$ pour tout $x$ et
le sous-graphe $G_M$ engendré par $M=\{x/x\in X, d_G(x)=q\}$ a pour chacun de ses blocs soit une clique
soit un cycle impair sans cordes.
\end{theorem}

\begin{theorem}[Dirac, 1952]
Si $G$ est un graphe de nombre chromatique $\chi(G)=q+1$, sans cliques de $q+1$ éléments, et
si l'on pose $S=\{x / d_G(x)>q\}$, alors on a :
$$\sum_{x \in S} (d_G(x)-q) \geq q-2$$
\end{theorem}

\begin{corollaire}
Si $G$ est $\chi$-critique avec $\chi(G)=q+1$ et sans cliques de $(q+1)$ éléments, alors le nombre $n$
de sommets et le nombre $m$ des arêtes vérifient :
$$2m\geq (n+1)q-2$$
\end{corollaire}
\setcounter{corollaire}{0}

\subsubsection{Construction de Haj\'{o}s}
\begin{defin}[Contraction élémentaire]
Soit $G$ un graphe simple ; on appelle contraction élémentaire sur $G$ tout opération qui
consiste à retirer deux sommets adjacents $a$ et $b$ de $G$ et à ajouter un sommet $c$ que
l'on relie à tous les sommets de $\Gamma_G(a)\cup\Gamma_G(b)$
\end{defin}

\begin{conj}[Hadwiger, 1943]
Tout graphe $G$ connexe avec $\chi(G)=q$ peut devenir un graphe complet à $q$ sommets au
moyen de contractions élémentaires.
\end{conj}

\paragraph{Transformation de Haj\'{o}s}$ $\\
Permet de construire à partir d'une $(q+1)$-clique tous les graphes qui ne sont pas $q$-chromatiques.

Soit $\mathcal{G}_q$ la classe de graphes tels que $\chi(G)>q$. On considère les trois
opérations suivantes sur $\mathcal{G}_q$ :
\begin{enumerate}
 \item\label{op1} On ajoute des arêtes et des sommets à $G\in\mathcal{G}_q$.
 \item\label{op2} Soient $G_1$ et $G_2$ deux graphes disjoints de $\mathcal{G}_q$, soient $a_1$, $b_1$ deux
sommets adjacents dans $G_1$ et soient $a_2$, $b_2$ adjacents dans $G_2$ ; on enlève l'arête $[a_1, b_1]$
dans $G_1$, l'arête $[a_2,b_2]$ dans $G_2$, on identifie $a_1$ et $a_2$, et l'on joint $b_1$ et $b_2$ par
une arête.
 \item\label{op3} On contracte en un seul point deux sommets non adjacents du graphe $G \in \mathcal{G}_q$.
\end{enumerate}

\begin{theorem}[Haj\'{o}s, 1961]
Tout graphe $G$ tel que $\chi(G)>q$ peut être obtenu à partir de la $(q+1)$-clique $K_{q+1}$ à
l'aide des opérations \ref{op1}, \ref{op2} et \ref{op3}.
\end{theorem}

\subsubsection{Dénombrement des colorations : polynômes chromatiques}
\setcounter{prop}{0}
$G=(X,E)$ un graphe, $x_1,\dots,x_n$ ses sommets.

\begin{defin}[Nombre de $\lambda$-colorations]
Nombre d'application $f(x) : X \longrightarrow \{1,2,\dots,\lambda\}$ telles que :
\begin{eqnarray*}
[x,y]\in E & \Rightarrow & f(x)\neq f(y)
\end{eqnarray*}
\end{defin}

\begin{defin}[Polynôme chromatique de $G$ en $\lambda$]
Fonction $P(G; \lambda)$ qui exprime le nombre des $\lambda$-colorations de $G$.
\end{defin}

\begin{rmq}
On désigne par $[\lambda]_n$ : $\lambda(\lambda-1)\dots(\lambda-n+1)$.
\end{rmq}

\begin{prop}
Soit $a$, $b$ deux sommets non adjacents du graphe $G$, soit $\tilde G$ le
graphe obtenu à partir de $G$ en reliant $a$ et $b$ par une nouvelle arête, et
soit $\bar G$ le graphe obtenu à partir de $G$ en contractant $\{a,b\}$. On a :
$$P(G;\lambda)=P(\tilde G;\lambda)+P(\bar G;\lambda)$$
\end{prop}

\begin{corollaire}
Si $G$ est un graphe d'ordre $n$, la fonction $P(G;\lambda)$ est un polynôme de degré $n$
en $\lambda$ ; en outre, le terme en $\lambda^n$ a pour coefficient 1 et le terme
constant est nul.
\end{corollaire}
\setcounter{corollaire}{0}


\begin{prop}
Si un graphe $G$ admet $p$ composantes connexes $H_1,\dots,H_p$, on a :
$$P(G;\lambda)=\prod_{i=1}^p P(H_i;\lambda)$$
\end{prop}

\begin{prop}
Si un graphe $G$ admet un ensemble d'articulations $A$ qui est une $k$-clique,
avec $q$ pièces $H_1,\dots,H_q$ relativement à $A$, alors :
$$P(G,\lambda)=([\lambda]_k)^{1-q}\prod_{i=1}^q P(H_i;\lambda)$$
\end{prop}

\begin{theorem}
Les coefficients de $P(G;\lambda)$ sont alternativement $\geq 0$ et $\leq 0$.
\end{theorem}

\begin{corollaire}
Si $G$ est un graphe d'ordre $n$ avec $m$ arêtes, le coefficient de
$\lambda^{n-1}$ est $-m$.
\end{corollaire}

\begin{theorem}
Un graphe $G$ d'ordre $n$ est un arbre si et seulement si $P(G;\lambda)=\lambda(\lambda-1)^{n-1}$
\end{theorem}

\subsubsection{Coloration de graphes planaires}
Résultats :
\begin{enumerate}
 \item Un graphe planaire qui ne contient pas quatre cycles de longueur 3 est coloriable avec 3 couleurs (Grünbaum, 1963).
 \item Un graphe planaire sans cycles de longueur 3 est coloriable avec 3 couleurs (Grötzsch, 1958).
\end{enumerate}

\begin{theorem}
Si $G$ est un graphe planaire, alors $\chi(G)\leq 5$.
\end{theorem}

\begin{theorem}[des quatre couleurs]
Si $G$ est un graphe planaire, alors $\chi(G)\leq 4$.
\end{theorem}

\begin{theorem}
Si $G$ est un graphe simple planaire à faces triangulaires et si les degrés sont tous des multiples de 2
(ou tous des multiples de 3), alors $\chi(G)\leq 4$.
\end{theorem}

\subsection{Étude spécifique}
Référence : cf \cite{jensen1996graph}\\
$G=(V,E)$

\begin{defin}[Coloring number]
$col(G)$ is the smallest number $d$ s.t. for some linear ordering $<$ of
the vertex set ; the ``back degree'' $|\{y:y<x,xy\in E(G)\}|$ of every
vertex $x$ is strictly less than $d$.\\
i.e. if $V(G)=\{x_1,\dots,x_n\}$, then :
$$col(G)=1+\underset{p}{min}\underset{i}{max}\{d(x_{p(i)},G_{p(i)}\}$$
where the minimum is taken over all permutations $p$ of $\{1,2,\dots,ni\}$,
and $G_{p(i)}$ is the subgraph induced by $\{x_{p(1)},\dots,x_{p(i)}\}$,
and where $d(x,H)$ denotes the degree of a vertex $x$ in a graph $H$.
\end{defin}

The coloring number can be computed in polynomial time.

\begin{theorem}[Hajnal, Szemerédi, 1970]
A graph $G$ may be colored by $\Delta(G)+1$ colors s.t. for any 2 colors i and j, where
$1 \leq i \leq j \leq \Delta (G) + 1$, the numbers of vertices of color $i$ and color $j$
differ by at most one.
\end{theorem}

\begin{theorem}[König, {[1916,1936]}]
Un graphe est 2-colorable ssi il n'admet pas de cycle impair.
\end{theorem}

\begin{theorem}[Dirac, {[1957]}]
Soit $G$ un graphe k-critique. Si $k \geq 4$ et $G \neq K_k$, alors :
$$2|E(G)| \geq (k-1)|V(G)|+(k-3)$$
\end{theorem}

\textbf{Le nombre chromatique d'un graphe parfait peut être calculé en temps polynomial
(Grötschel, Lov\'{a}sz, Schrijver, 1981).}\\
MAIS :\\
EST\_PARFAIT $\in$ \textit{\textbf{co-NP}} et on ne sait pas si EST\_PARFAIT appartient au moins à \textit{\textbf{NP}}
(cf \cite{jensen1996graph} p142).

\begin{theorem}[Perfect Graph Theorem. {[Lovász, 1972]}]
A graph is perfect if and only if its complement is perfect.
\end{theorem}

\begin{defin}
An induced cycle of odd length at least 5 is called an \textbf{odd hole}. An induced subgraph that is the complement of
an odd hole is called an \textbf{odd antihole}. A graph that does
not contain any odd holes or odd antiholes is called a \textbf{Berge graph}.
\end{defin}

\begin{theorem}[Strong Perfect Graph Theorem. {[Maria Chudnovsky, Neil Robertson, Paul D. Seymour, Robin Thomas]}]
A graph is perfect if and only if it is a Berge graph (i.e. ni lui ni son complémentaire
ne contiennent de cycle impair induit de longueur au moins cinq).
\end{theorem}

\begin{theorem}[Minty, 1962]
A graph $G$ is $k$-colorable if and only if $G$ has an orientation in which the flow ratio of any cycle $C$ (i.e.
the maximum of $m/n$ and $n/m$, where $n$ is the number of edges of $C$ pointing in one direction and $m$
is the number of edges of $C$ pointing in the opposite direction is at most $k-1$.
\end{theorem}

\begin{theorem}[Roy {[1967]}, Gallai {[1968]}]
A graph $G$ is $k$-colorable if and only if $G$ has an orientation in which the length of every directed path is
at most $k-1$.
\end{theorem}

There exists a function $g$ and a polynomial alorithm that for any given input graph $G$ will find
a number $s$, s.t. the $s\leq\chi(G)\leq g(s)$.\\
Proven by Alon, 1993 by replacing $\chi(G)$ by the list-chromatic number $\chi_l(G)$













