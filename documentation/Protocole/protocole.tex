\section{Généralités}
L'objectif de ce document est de décrire la démarche que
je vais suivre au long du TER.\\

On se place dans un multigraphe $G=(V,E)$. Pour le problème de coloration des
arrêtes, on considérera $G$ sans boucles (s'il y a une boucle, l'arrête est
adjacente avec elle-même et n'admet donc pas de bonne coloration) et dans le
cas de la coloration des sommets, on pourra considérer que $G$ est un graphe simple
(la multiplicité d'une paire de sommets quelconque ne change rien à la coloration).
On pourra également restreindre l'étude aux graphes à une composante connexe car
l'identification des composantes connexes se fait en $O(|V| + |E|)$ et le problème est
ensuite facilement parallélisable.\\

Il existe deux types de colorations différentes : la coloration des arrêtes et la
coloration des sommets.\\
L'objectif de ce TER est d'étudier la coloration des sommets. Si le temps le permet, on pourra peut-être ensuite essayer de ramener le problème de coloration des arrêtes à un problème de coloration des sommets.

\section{État de l'art}
La première partie de l'étude consistera à étudier l'état
de l'art en matière de coloration de graphes.\\
Lors de cette étude, je m'intéresserai dans un premier temps à
l'étude théorique et à des résultats généraux puis je rechercherai
les algorithmes existants.\\

J'essaierai également d'intuiter quelque
pistes d'algorithmes et de les décrire ou les écrire en pseudo-code.

\section{Coloration des sommets}

\subsection{Formulation du problème}
Formuler le problème : réfléchir aux différentes modélisations possibles.\\
Formuler en PLNE et éventuellement autrement.

\subsection{Résolution par PLNE}
Il s'agira ici de résoudre les problèmes de coloration par PLNE grâce à
des logiciels disponibles.

% \subsection{Autres méthodes exactes / Heuristiques}
% Il est difficile de savoir à quoi s'attendre en temps de
% calcul pour l'exécution d'un algorithme. Afin de pouvoir
% me donner une ``borne inf'' du temps de calcul nécessaire pour
% l'exécution d'un algorithme, je pense commencer
% par implémenter un algorithme approché intéressant. Ceci devra me permettre
% de comparer l'efficacité des algorithmes et d'éviter de lancer un calcul exact
% sur une instance pour laquelle un algorithme approché met déjà beaucoup de temps.

\subsection{Algorithmes}
S'il peut-être intéressant de résoudre le problème autrement que par PLNE,
il s'agira d'implémenter (ou éventuellement d'utiliser des
API disponibles) et de benchmarker des algorithmes exacts permettant
de colorer les sommets d'un graphe avec un nombre de couleur minimal (égal au nombre
chromatique $\chi (G)$).\\
Si le temps le permet, je m'intéresserai ensuite à des heuristiques.\\

Pour chaque algorithme j'adopterai la démarche suivante :
\begin{enumerate}
 \item Écrire l'algorithme en pseudo-code
 \item Étude préliminaire : réfléchir aux propriétés de l'algorithme :
       dans quelles instances sera-t-il
       le plus/le moins efficace, complexité, stabilité, à quel comportement s'attendre ?
       l'algorithme est-il garantit ?...
 \item Implémentation
 \item \label{bench} Benchmark
 \item Questionnement / Analyse des problèmes : les résultats correspondent-ils
       à ceux attendus ? si non pourquoi ?
       le temps d'exécution est-il satisfaisant ? peut-on l'améliorer ?
 \item Amélioration, optimisation
 \item fin ou retour en \ref{bench} si l'algorithme a été amélioré
\end{enumerate}

Je m'attacherai lors de cette partie à regarder quels algorithmes sont efficaces dans quels graphes
et s'il est intéressant de rechercher des propriétés du graphe (recherche qui peut être coûteuse)
afin de déterminer quel algorithme lancer ou de faciliter l'exécution d'un algorithme.\\

Pour ce qui est des benchmark, je commencerai par exécuter les algorithmes sur de petites instances
afin de pouvoir vérifier à la main la bonne coloration (pour le vérifier sur des instances de plus
grande taille j'implémenterai peut-être un algorithme). Je les exécuterai ensuite sur des graphes ayant certaines propriétés et enfin, je les exécuterai sur des graphes de
plus grande taille pour effectuer des tests de performances.\\
Afin de pouvoir comparer les algorithmes, je créerai probablement un jeu de
tests composé de graphes de grandes tailles
incluant des graphes sans propriétés remarquables et d'autres avec.

% \section{Application des méthodes de coloration des sommets aux arrêtes}
% Je rechercherai comment appliquer les méthodes de coloration des sommets à la coloration des arrêtes.\\
% 
% Idées : graphes de lignes, matrice d'adjacence...

\section{Pour aller plus loin...}
Si le temps le permet je pourrais envisager d'autres approches du problème et suivre d'autres pistes
telles que la parallélisation (est-ce possible, si oui le faire), l'écriture d'algorithmes probabilistes,
la réécriture d'un algorithme dans un autre langage de programmation...\\
Je pourrais éventuellement essayer d'appliquer les méthodes de coloration des sommets à
la coloration des arrêtes à l'aide des graphes de lignes par exemple.\\

D'autres idées viendront probablement étoffer cette partie au cours du TER. 

\end{document}
